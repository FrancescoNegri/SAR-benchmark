\documentclass[12pt]{article}
\usepackage{amsmath}
\usepackage{graphicx}
\usepackage{hyperref}
\usepackage{mathtools}
\usepackage[latin1]{inputenc}
\usepackage{cancel}
\usepackage[margin=2.5cm]{geometry}
\usepackage{float}
\setlength\parindent{0pt}

\title{Stimulus Artifacts Rejection Benchmarking Tool}
\author{Francesco Negri}
\date{May 2023}

\begin{document}
\maketitle

% \tableofcontents
% \newpage

\section*{Assumptions}
The following work about Stimulus Artifacts Rejection from electrophysiological data
recorded via the Intan system in in-vivo animals relies on a number of crucial assumptions.
\begin{enumerate}
      \item A single stimulus is made by two parts with different dynamics:
            \begin{enumerate}
                  \item A short-lasting saturation of the amplifier described by a peak.
                  \item A long-lasting transient phase with a quasi-exponential behaviour, due
                        to the discharge of amplifier's capacitors.
            \end{enumerate}
      \item Stimulation occurs exclusively in one \textit{Channel} for each \textit{Animal}.
      \item Each \textit{Tank \(>\) Animal \(>\) Block} is characterized by
            the very same sequence of stimuli.
      \item Each \textit{Tank \(>\) Animal \(>\) Block \(>\) Channel} is expected to
            always display the same response over time, as the shape of the stimulus depends
            on the geometry between the observed channel and the stimulated one. The geometry
            might also affect the polarity of the stimulus. \textbf{[REVIEW]}
      \item The baseline activity observed in a \textit{Channel} tend to remain in the
            same range during the whole recording.
\end{enumerate}

\section*{Synthetic Dataset Generation}
This section aims at providing the reader with an extensive understanding of
the steps followed to generate a synthetic dataset to test Stimulus Artifact Rejection
algorithms against.\\
A synthetic dataset has been preferred with respect to considering
actual data as it enables a full knowledge of the clean signal, allowing to compute
performance metrics and compare the different algorithms. In addition, a data-driven
approach was employed, since it might simulate real situations in a better way than
a model-based approach, which requires a higher number of assumptions with the increased
risk of introducing bises.

\subsection*{Extraction of Artifact Templates}

\subsection*{Extraction of Baseline Activity}

\subsection*{Generation of Snippets}
3xN arrays:
\begin{itemize}
      \item First row: stimulus train (0s and 1s)
      \item Second row: synthetic data
      \item Third row: clean data to compute metrics
\end{itemize}

\section*{The Benchmarking Tool}

\end{document}