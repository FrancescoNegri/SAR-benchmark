\documentclass[12pt]{article}
\usepackage{amsmath}
\usepackage{graphicx}
\usepackage{hyperref}
\usepackage{mathtools}
\usepackage[latin1]{inputenc}
\usepackage{cancel}
\usepackage[margin=2.5cm]{geometry}
\usepackage{float}
\setlength\parindent{0pt}

\title{Stimulus Artifacts Rejection Benchmark Tool}
\author{Francesco Negri}
\date{May 2023}

\begin{document}
\maketitle

% \tableofcontents
% \newpage

\section*{Assumptions}
The following work about Stimulus Artifacts Rejection from electrophysiological data
recorded via the Intan system in in-vivo animals relies on a number of crucial assumptions.
\begin{enumerate}
    \item A single stimulus is made by two parts with different dynamics:
          \begin{enumerate}
              \item A short-lasting saturation of the amplifier described by a peak.
              \item A long-lasting transient phase with a quasi-exponential behaviour, due
                    to the discharge of amplifier's capacitors.
          \end{enumerate}
    \item Stimulation occurs exclusively in 1 \textit{Channel} for each \textit{Animal}.
    \item Each \textit{Tank \(>\) Animal \(>\) Block} is characterized by
          the very same sequence of stimuli.
    \item Each \textit{Tank \(>\) Animal \(>\) Block \(>\) Channel} is expected to
          display always the same response over time, as the shape of the stimulus depends
          on the geometry between the observed channel and the stimulated one. The geometry
          might also affect the polarity of the stimulus.
\end{enumerate}

\end{document}